% -*- latex -*-

% This file can be used for troubleshooting problems.

% Adjust this file to your liking and then build using:
%    make mathics-test.pdf
\chapter{Test Chapter}
\chapterstart
\chaptersections
\section*{TestSection}
\sectionstart

Here is what we want to test:

\begin{asy}
if(!settings.multipleView) settings.batchView=false;
settings.tex="xelatex";
defaultfilename="mathics-test-1";   % Note that filename line has been altered!
if(settings.render < 0) settings.render=4;
settings.outformat="";
settings.inlineimage=true;
settings.embed=true;
settings.toolbar=false;
viewportmargin=(2,2);

% Copy stuff here from mathics-xxx.asy
% The below is from the "Darker" section around mathics-114.asy

import three;
import solids;
size(6.6667cm, 6.6667cm);
currentprojection=perspective(2.6,-4.8,4.0);
currentlight=light(rgb(0.5,0.5,0.5), specular=red, (2,0,2), (2,2,2), (0,2,2));
// Sphere3DBox
draw(surface(sphere((0, 0, 0), 1)), rgb(0.0,0.6666666666666667,0.0));
draw(((-1,-1,-1)--(1,-1,-1)), rgb(0.4, 0.4, 0.4)+linewidth(1));
draw(((-1,1,-1)--(1,1,-1)), rgb(0.4, 0.4, 0.4)+linewidth(1));
draw(((-1,-1,1)--(1,-1,1)), rgb(0.4, 0.4, 0.4)+linewidth(1));
draw(((-1,1,1)--(1,1,1)), rgb(0.4, 0.4, 0.4)+linewidth(1));
draw(((-1,-1,-1)--(-1,1,-1)), rgb(0.4, 0.4, 0.4)+linewidth(1));
draw(((1,-1,-1)--(1,1,-1)), rgb(0.4, 0.4, 0.4)+linewidth(1));
draw(((-1,-1,1)--(-1,1,1)), rgb(0.4, 0.4, 0.4)+linewidth(1));
draw(((1,-1,1)--(1,1,1)), rgb(0.4, 0.4, 0.4)+linewidth(1));
draw(((-1,-1,-1)--(-1,-1,1)), rgb(0.4, 0.4, 0.4)+linewidth(1));
draw(((1,-1,-1)--(1,-1,1)), rgb(0.4, 0.4, 0.4)+linewidth(1));
draw(((-1,1,-1)--(-1,1,1)), rgb(0.4, 0.4, 0.4)+linewidth(1));
draw(((1,1,-1)--(1,1,1)), rgb(0.4, 0.4, 0.4)+linewidth(1));
\end{asy}
\sectionend
\chapterend
